
		\HgraficarEPS{0.60}{graf_espectro_env}{Envolventes del filtro LPC a lo largo del tiempo.}{fig:espectro_env}

		La Figura \ref{fig:espectro_env} consta de las curvas envolventes a la largo del tiempo y la Figura \ref{fig:specgram} es el espectograma de la señal original realizado con la función \texttt{specgram()}. Se puede ver que ambos gráficos son muy similares, salvando la diferencia que el de envolventes es más suave pudiendose ver mejor los formantes de cada letra. Sin embargo, el gráfico del \texttt{specgram()} tiene como ventaja que se puede localizar mejor cuándo sucedio cada vocal. Ésto puede deverse al tamaño de ventana que usa la función por omisión.
		\HgraficarEPS{0.6}{graf_specgram}{Espectrograma de la señal ventaneada.}{fig:specgram}
