
	Al calcular la respuesta en frecuencia, se ve en la Figura \ref{fig:envolvente} que la reconstrucción sigue con claridad a la envolvente de la original. Con dicho gráfico se puede afirmar que el primer formante se encuentra cerca de los \SI{750}{\Hz} y el segundo, \SI{1800}{\Hz}.
	
		\graficarEPS{0.7}{graf_envolvente}{Comparación en frecuencia de las señal original y LPC.}{fig:envolvente}

	\pagebreak
	En la Figura \ref{fig:vocales} se pueden ver las envolventes para las vocales de la señal de audio \emph{fantasía}. Del gráfico se destaca que los formantes de la letra \emph{a} se agrupan cerca de los \SI{800}{\Hz} y \SI{1600}{\Hz} mientras que la \emph{í} tiene un formante de baja frecuencia (\SI{350}{\Hz}) y otro más alto (\SI{3600}{\Hz}), pudiendose así distinguir claramente a partir del análisis de envolventes las vocales.
		\graficarEPS{0.65}{graf_vocales}{Comparación en frecuencia de las señal original y LPC.}{fig:vocales}
