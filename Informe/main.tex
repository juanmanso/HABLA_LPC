\documentclass[10pt,a4paper]{article}

\usepackage[utf8]{inputenc}		% Configuro la codificación
\input{config.tex}				% Archivo con los comandos globales como Título y autores
\input{preamble.tex}
\input{aux_functions.tex}		% Se proveen un conjunto de funciones extras

% Defino el path de los includegraphics
\graphicspath{{./Figuras/}}		% Directorio que contiene los graficos

% Defino el path para los input de .tex y de .eps
\makeatletter
\def\input@path{{./Figuras/}{./Secciones/}{./Cover_page/}}
\makeatother

% Defino el path del listings
\ifListings
%% Cambiar el nombre de la carpeta si se utilizan Listings
	\lstinputpath{{../Octave/}}
\fi

\definecolor{myred}{rgb}{0.5,0,0}
\definecolor{mygreen}{rgb}{0,0.5,0}

\renewcommand{\thesubsection}{\thesection.\alph{subsection}}

\begin{document}
		% Carátula (formal o simple,_formal o _simple respectivamente) con Resumen
		% incluido e Índice (si es necesario configurar en config.tex) del informe
		\input{cover_formal.tex}
	\setcounter{page}{1}

	\part{Caracterización de la señal}
		A continuación se realiza el cálculo de los coeficientes del filtro del tracto vocal y a partir del mismo caracterizar la señal entrante.

%		\graficarPNG{bgraf_og_vs_recon}{Señal original versus la reconstrucción por LPC.}{fig:ogrecon}
%		\graficarEPS{0.7}{graf_og_vs_recon}{Señal original versus la reconstrucción por LPC.}{fig:ogrecon}
%		\graficarEPS{0.7}{graf_prueba}{Señal original versus la reconstrucción por LPC.}{fig:ogrecon}
	
		\section{Reconstrucción de una ventana}
			
	La señal a caracterizar es procesada bajo el algoritmo de LPC, iterando con muestras de \SI{0.025}{\s} para generar los coeficientes LPC que se la asignan a los primeros \SI{.01}{\s}. La función LPC fue implementada en \emph{Octave} de la siguiente manera:
	\lstinputlisting{funcionlpc.m}

	Tras el cálculo de coeficientes, utilizando la función \texttt{filter()} se reconstruye la señal ventaneada. En la Figura \ref{fig:ogrecon} se exponen la señal original y reconstruida por LPC.

		\graficarEPS{0.7}{graf_og_vs_recon}{Señal original versus la reconstrucción por LPC.}{fig:ogrecon}

 Al igual que en la Figura \ref{fig:ogrecon}, se puede ver en la Figura \ref{fig:err} que hay pequeñas diferencias cerca de los picos de la curva pero en general sigue a la original con precisión (difiere como mucho en \num{0.09}). Otra característica importante que revela el análisis del error en la Figura \ref{fig:err} es la frecuencia glótica, dado que se supuso una entrada de ruido blanco y por tanto (al estar relacionado con el pulso glótico) se expone como error.
		\graficarEPS{0.7}{graf_err}{Error de estimación de la reconstrucción por LPC.}{fig:err}
%	\lstinputlisting[linerange=Resultados- ]{ej_3.m}


		\section{Análisis de envolventes}
			
	Al calcular la respuesta en frecuencia, se ve en la Figura \ref{fig:envolvente} que la reconstrucción sigue con claridad a la envolvente de la original. Con dicho gráfico se puede afirmar que el primer formante se encuentra cerca de los \SI{750}{\Hz} y el segundo, \SI{1800}{\Hz}.
	
		\graficarEPS{0.7}{graf_envolvente}{Comparación en frecuencia de las señal original y LPC.}{fig:envolvente}

	\pagebreak
	En la Figura \ref{fig:vocales} se pueden ver las envolventes para las vocales de la señal de audio \emph{fantasía}. Del gráfico se destaca que los formantes de la letra \emph{a} se agrupan cerca de los \SI{800}{\Hz} y \SI{1600}{\Hz} mientras que la \emph{í} tiene un formante de baja frecuencia (\SI{350}{\Hz}) y otro más alto (\SI{3600}{\Hz}), pudiendose así distinguir claramente a partir del análisis de envolventes las vocales.
		\graficarEPS{0.65}{graf_vocales}{Comparación en frecuencia de las señal original y LPC.}{fig:vocales}


			\pagebreak
		\section{Espectograma}
			
		\HgraficarEPS{0.60}{graf_espectro_env}{Envolventes del filtro LPC a lo largo del tiempo.}{fig:espectro_env}

		La Figura \ref{fig:espectro_env} consta de las curvas envolventes a la largo del tiempo y la Figura \ref{fig:specgram} es el espectograma de la señal original realizado con la función \texttt{specgram()}. Se puede ver que ambos gráficos son muy similares, salvando la diferencia que el de envolventes es más suave pudiendose ver mejor los formantes de cada letra. Sin embargo, el gráfico del \texttt{specgram()} tiene como ventaja que se puede localizar mejor cuándo sucedio cada vocal. Ésto puede deverse al tamaño de ventana que usa la función por omisión.
		\HgraficarEPS{0.6}{graf_specgram}{Espectrograma de la señal ventaneada.}{fig:specgram}


	\part{Codificación}
		El objetivo del algortimo de \texttt{LPC} es la codificación de la señal de habla para facilitar su transimisión y recepción. Para ello se realiza el cálculo de coeficientes del filtro resultante de la modulación de las articulaciones. Con ésto y el error, se puede reconstruir la señal con cierta incertidumbre establecida por la precisión utilizada.

		Cabe acalarar que el \emph{script} \texttt{ej3.m} genera gráficos que no permiten ver con claridad el efecto de las operaciones realizadas. Es por ésto que las comparaciones y gráficos se generan con la ventana previamente analizada.
		
		\section{Reconstrucción con redondeo}
			
	La función de redondeo fue implementada de la siguiente manera:
		\lstinputlisting{redondear.m}

	A continuación se muestran los gráficos comparativos entre las distintas reconstrucciones. 
		\HgraficarEPS{0.7}{graf_ogvsrecon_redon}{Comparación entre la señal original y las distintas reconstrucciones.}{fig:ogrecon_redon}

	Se puede ver en la Figura \ref{fig:ogrecon_redon} que la señal reconstruida con 8 bits tiene un error apreciable en los picos de la señal. Al analizar la Figura \ref{fig:err_redon} se comprueba que el error de la codificación en 8 bits es apreciable pero muy pequeña, estando por debajo de \num{.03}.

	Lo curioso es que la señal reconstruida sin redondeo presenta un error muy grande (llegando cerca de \num{.8}) cuando la misma debería ser la reconstrucción más precisa. Se puede ver que la curva del error se asemeja a la señal de error.
		\HgraficarEPS{0.7}{graf_err_redon}{Comparación de las incertidumbres entre las distintas reconstrucciones.}{fig:err_redon}


		\section{Compresión de la señal}
			
	El archivo de audio \emph{fantasia.wav} requirió en memoria \num{348800} bytes. Por lo tanto si se realizase una transmisión de dicho archivo completo, será muy costoso.

	Si en cambio se utilizase una codificación de 8 bits y se enviasen los coeficientes LPC y la señal de error, el paquete pesaría:
	\begin{align*}
		{Memoria_{LPC}} &= {size}({LPC_{coef}}) + {size}(err8)\\
		&= \num{41192}{ bytes} + \num{3208}{ bytes} = \num{44400}{ bytes}
	\end{align*}

	De esta manera se logra comprimir un archivo de \num{348800} bytes a uno de \num{44400} (se comprimió a un 12\%).

	

	\part{Conclusiones}\label{part:conclusiones}
		
	Se pudo ver que la técnica LPC es muy útil tanto para el análisis de señales de habla como para su codificación. Con ella se pueden distinguir vocales a partir del análisis de formantes, como también detectar la variación de la frecuencia glótica a traves de las envolventes en el tiempo. Como herramienta de compresión es muy poderosa porque puede lograr disminuir el tamaño en un orden de magnitud con una precisión muy alta.


	% \appendix
\end{document}
